\begin{abstract}

\setcounter{page}{5}

Отчет 78 страниц, 31 рисунок, 20 таблиц, 13 источников.

ИНТЕРНЕТ-ПРОВАЙДЕР, ГЛУБОКИЙ АНАЛИЗ ПАКЕТА, СЕТИ, MPLS, VLAN, МЕЖСЕТЕВОЙ ЭКРАН, МЕТКА, ТЕГИРОВАНИЕ, ПРОТОКОЛЫ, ОБНАРУЖЕНИЕ, DPDK.

В данной дипломной работе рассмотрены вопросы проектирования и разработки сетевого сервиса классификации трафика на основе технологии DPI.

В аналитическом разделе подробно рассмотрена предметная область, проведен сравнительный анализ существующих аналогов разрабатываемого ПО и сформулированы требования к разрабатываемому сервису. Также приведены возможности и основные особенности фреймворков быстрой обработки пакетов PF\_RING и DPDK.

В конструкторском разделе описана структура каждого протокола, поддерживаемого сервисом. Разработаны алгоритмы обнаружения и приведены их блок-схемы. Также описана структура и допустимые значения файла конфигурации.

В технологическом разделе описан выбор средств и технологии разработки, особенности, используемые при разработке данного ПО. Описан процесс установки и запуска приложения, допустимые ключи и их значения. Также представлены результаты тестирования.

В организационно-экономическом разделе приведены расчеты по определению структуры затрат на разработку проекта, а также выполнено планирование цены программного продукта и прогнозирование прибыли.

В разделе охраны труда и экологии рассмотрены опасные и вредные факторы, влияющие на программиста при разработке, а также приведен расчет уровня шума в серверной комнате.

В заключении делается вывод о результатах, достигнутых в ходе выполнения дипломной работы.
\end{abstract}
